%%
%% Author: Deeps
%% 22.06.2018
%%

% Preamble
\documentclass[
a4paper,		% Verwenden von A4 als Seitenformat
12pt,			% Festlegung der Standardtextgröße auf 12pt
pagesize,
headsepline,
titlepage		% Erstellen einer Titelseite, Automatisches zentrieren der Titelseite
]{scrartcl}

% Packages
\usepackage{comment}            % Kommentarblöcke

\usepackage[ngerman]{babel}		% deutsche Trennmuster
\usepackage[utf8]{inputenc}		% direkte Eingabe von Umlauten & Co. (Vorsicht: Encoding im Editor muss auch UTF-8 sein!)

\usepackage[T1]{fontenc}			% T1-Schriften

\usepackage{mathptmx}			% Times/Mathe \rmdefault
\usepackage[scaled=.90]{helvet}	% Skalierte Helvetica \sfdefault
\usepackage{courier}			% Courier \ttdefault

\usepackage{
amsmath,	% Mehr mathematische Symbole
amsthm,		% Mehr mathematische Symbole
amsfonts,	% Mehr mathematische Symbole
graphicx, 	% Einbindung von Grafiken
caption 		% Bildunterschriften
}

% Wenn man direkt mit dem pdflatex eine PDF-Datei erzeugt, sollten diese beiden Pakete eingebunden werden
\usepackage{hyperref} % Hyperlinks anklickbar
\usepackage{ae, aecompl} % bessere Bildschirmschriftarten

% Code anzeigen
\usepackage{listings}
\usepackage{algorithm}
\usepackage{algpseudocode}

\usepackage{natbib}

\usepackage{url}        %bibtex
\usepackage{graphicx}

% Styling
\pagestyle{headings}

\headsep4mm % Abstand der Kopfzeile vom Text:

\typearea[current]{current}     % Satzspiegel neu berechnen

\titlehead{
\vspace*{2cm}
\centering \includegraphics[height=3cm]{resources/hpi-logo.pdf}
}
\subject{Bachelorarbeit}
\title{
Maschinelles Lernen im Onlinehandel: \\
Extraktion Produktspezifischer Daten \\
\bigskip
\large{Content Extraction from Web Pages Using Machine Learning}
\medskip ~\\
}
\author{
Leonardo Hübscher\\
\\B.Sc.\\
IT-Systems Engineering\\
Fachgebiet für Informationssysteme \\
\bigskip ~\\
Betreuer:\\
Prof. Felix Naumann\\
Leon Bornemann\\
Stanislav Nowogrudski
}
\date{20. Juli 2018}

% Document
\begin{document}

    \maketitle

    \begin{abstract}
        \section*{Zusammenfassung}
\markboth{Zusammenfassung}{Zusammenfassung}
\label{sec:abstract}
    \end{abstract}

    \tableofcontents
    \newpage

    \section{Die Welt der Preisvergleichsportale}
\label{sec:einleitung}

Der Fernhandel ist bereits seit der Steinzeit ein wichtiger Bestandteil der Gesellschaft.
Durch die rasante Entwicklung des Internets und der steigenden Anzahl der Onlinehändler vergrößert sich das
Produktangebot.\\
Heutzutage kann ein Käufer aus einer Vielzahl von Artikeln wählen und muss sich nicht wie in der Steinzeit auf
einen Händler oder auf die lokale Verfügbarkeit beschränken.

\subsection{Der Onlinehandel von heute}
\label{subsec:onlinehandel-heute}

In den letzten Jahren hat der Onlinehandel sowohl an Bedeutung für die Unternehmen, als auch für die Kunden gewonnen.
Laut einer Statistik von Eurostat machte im Jahr 2017 der Onlinehandel 21\% des Gesamtumsatzes deutscher Unternehmen
aus und stellte somit einen nicht unerheblichen Anteil dar~\cite{statista:anteil-gesamtumsatz-europa}.
Aus einer weiteren Statistik geht hervor, dass 2017 zwei Drittel der Deutschen regelmäßig online
einkauften~\cite{statista:anteil-online-kaeufer-europa}.

Die Entwicklung bringt jedoch ein Problem mit sich: Mit der steigenden Auswahl an Produkten verliert ein
potenzieller Käufer schnell die Übersicht.
Preisvergleichsportale versuchen deshalb die Markttransparenz wiederherzustellen.
Ein Käufer soll sich sicher sein das für ihn beste Angebot zu finden.

Das Angebot der Vergleichsportale wird von den Internetnutzern gut aufgenommen, wie eine Statistik aus 2017 zeigt:
Für einen besseren Vergleich nutzen rund zwei Drittel die Möglichkeit sich im Internet zum Produkt oder zum
Preis zu informieren~\cite{statista:internetnutzer-preisvergleich-deutschland,
statista:anteil-online-kaeufe-deutschland}.

In einem vom Nachrichtensender n-tv beauftragten Test\footnotemark, hat das Deutsche Institut für Service-Qualität
mehrere Preissuchmaschinen unter dem Aspekt des günstigsten Preises, der Preisaktualität und dem Nutzererlebnis
verglichen.
\footnotetext{https://disq.de/2014/20141004-Preissuchmaschinen.html}
\newline
Im Ergebnis hat idealo.de in allen Kategorien den ersten Platz eingenommen, gefolgt von billiger.de und preis.de.
Es wurde jedoch bemängelt, dass selbst beim besten Preisvergleich nur für die Hälfte der Produkte der günstigste
Shop angezeigt wurde.

\subsection{Das Preisvergleichsportal idealo}
\label{subsec:idealo}

Die Mission des Preisvergleichsportals idealo ist es, den Vergleich für den Kunden stetig zu verbessern.
Je mehr Angebote idealo vergleicht, desto sicherer kann sich der Kunde sein, das tatsächlich günstigste Angebot zu
finden.

Dazu schließt idealo Verträge mit mehreren Onlinehändlern ab.
Diese verpflichten sich, Daten zu ihren Angeboten an idealo zu übermitteln und zu aktualisieren.
Für jeden vermittelten Kauf zahlen die Shops an idealo eine Provision.
Diese basiert auf CPC (Kosten pro Klick) oder CPO (Kosten pro tatsächlicher Bestellung).

Um auch zukünftig wettbewerbsfähig zu bleiben arbeitet idealo daran, auch für die letzten 50\% der Produkte
immer das beste Angebot liefern zu können.
Dies erreicht es zum einen durch Vertragsabschlüsse mit weiteren Onlineshops und zum anderen durch die
Sicherstellung, dass tatsächlich alle Angebote eines Vertragspartners gelistet werden.

\subsection{Das Ziel des Bachelorprojektes}
\label{subsec:projektziel}

Es soll eine Software entworfen werden, welche eine automatisierte Bestandsanalyse für einen bestimmten
Vertragspartner durchführt.
Mit Hilfe des resultierenden Berichtes soll es möglich sein, herauszufinden welche Angebote des Vertragspartner im
Produktkatalog von idealo fehlen.
\\
~\\
Der Bericht soll Informationen darüber enthalten, welche Produkte nicht vorhanden sind, aus welcher Kategorie diese
stammen und zu welcher Preisregion die Produkte gehören.
Durch diese Übersicht soll ein Mitarbeiter von idealo dazu befähigt werden, die Ursachen für das Fehlen der Angebote
herauszufinden.

idealo vermutet, dass ein unbeabsichtigtes Fehlen von Produkten durch einen fehlerhaften Importvorgang zu erklären ist.
Zudem könnte es sein, dass ein Händler bewusst nicht alle Produkte bei idealo führen möchte.
\\
~\\
Für die Entwicklung dieser Lösung hatten wir als fünfköpfiges Team neun Monate Zeit.
Zudem wurde uns ein Betreuer von idealo zur Verfügung gestellt, welcher die funktionalen Anforderungen an die
Software kommunizierte und als technischer Berater diente.
Er begleitete uns während des gesamten Entwicklungsprozesses und unterstützte uns bei Fragen bezüglich der
Systemarchitektur.

\subsection{Die Microservice-Architektur des Scout-Softwaresystems}
\label{subsec:microservice-architektur}

Wir haben uns dafür entschieden, das Gesamtsystem als Microservice-Architektur zu konzipieren.

Die Mircoservice-Architektur ermöglicht es logisch gekapselte Komponenten zu entwickeln, welche sich sehr gut
skalieren und erweitern lassen.
Eine ausführlichere Begründung für diese Architekturentscheidung kann in der Bachelorarbeit von Dmitrii
Zhamanakov nachgeschlagen werden.

Für die Implementierung der Architektur haben wir die Programmiersprache Java gewählt und verwenden diese in
Kombination mit dem Spring-Framework\footnotemark.
\footnotetext{\url{https://spring.io/}}

Das entwickelte Gesamtsystem \textit{Scout} besteht grob gesehen aus drei Komponenten: dem Crawler, dem Parser und dem
Matcher.

Der \textit{Crawler} ist für das Herunterladen jeder einzelnen Seite eines Shops verantwortlich.
Jonas Pohlmann hat sich im Projektverlauf intensiv mit verschiedenen Crawling-Frameworks auseinandergesetzt und diese
in seiner Bachelorarbeit verglichen.

Die Funktionsweise der maschinenlernbasierten \textit{Matcher}-Komponente wird in der Bachelorarbeit von Tom
Schwarzburg näher beschrieben.
Der Matcher vergleicht die vom System gefundenen Angebote mit den Angeboten, die idealo bereits kennt.

Damit dieser die geladenen Angebote mit dem Katalog von idealo vergleichen kann, muss das heruntergeladene
HTML-Dokument in ein Format gebracht werden, welches der Computer für den Vergleich nutzen kann.

Diesen Schritt erledigt der Parser, welcher zwischen Crawler und Matcher agiert.
Der Fokus dieser Arbeit liegt in der Beschreibung der Funktionsweise und des Aufbaus des Parsers.
    \newpage
    \section{Extraktion produktspezifischer Daten}
\label{sec:extraktion-produktspezifischer-daten}

Das Extrahieren der Informationen ist ein sehr wichtiger und kritischer Schritt, da die Genauigkeit die Ergebnisse
des Matchers und somit auch des Gesamtergebnisses stark beeinflusst.
Fehlerhafte Informationen könnten das Ergebnis so stark verfälschen, dass die Aussagekraft des Gesamtergebnisses des
Systems in Frage gestellt werden könnte.

Die größte Schwierigkeit bei der Extraktion der produktspezifischen Daten besteht darin, dass der Computer nicht
weiß, an welchen Stellen er die gewünschten Informationen findet.
Um dieses Problem zu lösen, unterscheiden wir zwischen zwei Herangehensweisen, welche im nachfolgenden Kapitel
erläutert werden.

\subsection{Die möglichen Herangehensweisen}
\label{subsec:herangehensweisen}

\begin{comment}
    schema.org
\end{comment}

Es gibt grundsätzlich zwei Möglichkeiten, wie man das Problem der Datenextraktion angehen kann.
Vordergründig geht es hierbei um die Frage, wie man aus den unterschiedlich strukturierten Shops Informationen
extrahiert.

Das von uns zu lösende Problem gibt es nicht zum ersten Mal.
Bereits Google hat sich mit der Thematik befasst, da es bei den Suchergebnissen nähere Informationen zu den Angeboten
anzeigen möchte.
Im Zuge dessen hat Google gemeinsam mit den Suchmaschinen von Microsoft, Yahoo und Yandex den Schema.org - Standard für
die strukturierte Datenangabe im Internet entwickelt.
Laut einer Schätzung von idealo verwenden rund 40\% der Shops diesen Standard.
Diese Herangehensweise bezeichnen wir als shop-unspezifischen Ansatz, da wir eine generische Regel verwenden können,
um die standardisierten Informationen zu erfassen.
Dieser Option hat den klaren Vorteil, dass sie recht schnell und einfach umzusetzen ist.

Alternativ zum shop-unspezifischen Ansatz, gibt es die shop-spezifische Herangehensweise.
Bei dem shop-spezifischen Ansatz geht es darum, für jeden Shop gesonderte Spezifikationen zu erstellen.
Die Regeln des shop-unspezifischen Ansatzes würden eine Untermenge des shop-spezifischen Ansatzes darstellen.
Die Umsetzung dieser Variante wäre insgesamt anspruchsvoller.
Insgesamt gehen wir jedoch davon aus, sowohl die Extraktionsrate als auch die Präzision des Parsers im Vergleich zu
dem shop-unspezifischen Ansatz zu erhöhen und somit bessere Daten für das Vergleichen zu erhalten.

Zu Beginn haben wir erste Versuche basierend auf dem Schema.org-Standard unternommen.
Leider haben wir schnell feststellen müssen, dass die Spezifikation oft nicht richtig eingehalten wurden, was die
Datenqualität stark verringert hat.
Auch bei anderen Standards, welche bei der Strukturierung von Produktdaten im Internet helfen sollen wie zum Beispiel
JSON-LD (W3C) und das Open-Graph-Protokoll (Facebook) konnten wir die gleiche Beobachtung machen.
Wir haben uns deshalb gegen diesen Ansatz entschieden.

Im nachfolgenden wird darauf eingegangen, wie wir die shop-spezifische Methode umgesetzt und in den Ablauf der
Extraktionsprozesses integriert haben.

\subsection{Annahmen und Grundidee}
\label{subsec:annahmen-und-grundidee}

Wir folgen der Annahme, dass jeder Shop ein CMS (Content Management System) zur Verwaltung seiner Angebote verwendet.
Des Weiteren gehen wir davon aus, dass sich durch die Verwendung eines CMS die Struktur der Angebote ähnelt und diese
erlernt werden kann.
Des Weiteren nehmen wir an, dass idealo aufgrund der Vertragsvereinbarungen für den zu untersuchenden Shops bereits
eine gewisse Menge an Angeboten besitzt und die Produktattribute nicht manipuliert wurden.

Der grobe Ablauf der shop-spezifischen Datenextraktion kann in zwei Phasen untergliedert werden:
dem Generieren der shop-spezifischen Extraktionsregeln und dem Anwenden dieser auf die von der Crawler-Komponente
heruntergeladenen Seiten.
Unter der Regel kann man sich zunächst so etwas wie eine Wegbeschreibung durch das HTML-Dokument zu dem gewünschten
Element vorstellen.
Die Grundidee der ersten Phase besteht darin, dass die Regeln, welche für das Extrahieren benötigt werden mit
Hilfe der Daten von idealo angelernt werden können.
Die generierten Regeln werden nun in der zweiten Phase angewendet, sodass für jedes Attribut genau ein Wert
zugeordnet wird.
Für jede gecrawlte Seite werden die extrahierten Werte für jedes Attribut gemeinsam abgespeichert.

Die Logik der beiden Komponenten spiegelt sich auch in der genaueren Architektur des Parsers wieder.
Dieser besteht aus der Shop Rules Generator (SRG) Komponente und der Parser Komponente.
Im nachfolgenden werde ich auf die Funktionsweise der beiden Komponenten einzeln eingehen.

\subsection{Die Funktionsweise der Parser-Komponente}
\label{subsec:funktionsweise-parser}

Die Parser-Komponente lauscht auf eine Queue, über die sie die vom Crawler heruntergeladenen Seiten erhält.
Der Crawler sendet zu jeder Seite auch die Webadresse und die Identifikationsnummer des zugehörigen Shops mit.
Nachdem der Parser etwas aus der Queue empfangen hat, fragt er den SRG nach Regeln für den entsprechenden Shop.
Sollten die Regeln noch nicht existieren, so wartet er solange, bis der SRG diese erstellt hat.
Sobald dieser die Regeln als Antwort sendet, wendet der Parser diese Regeln auf der gecrawlten Seite an.
Der extrahierte Preis und die Bilderlinks werden von dem Parser normalisiert, ehe sie gemeinsam mit den anderen
Attributen als ParsedOffer in einer Datenbank persistiert werden.
Der Matcher greift später auf diese Datenbank für den Vergleich zu.

Um den nichtfunktionalen Anforderungen, insbesondere der Skalierbarkeit und Parallelität zu genügen, arbeitet die
Parser-Komponente mit mehreren Threads und kann somit mehrere gecrawlte Seiten gleichzeitig verarbeiten.
Zudem wurden Cache-Mechanismen verwendet, um die Anfragen an den SRG zu minimieren.


\subsection{Die Funktionsweise des SRG}
\label{subsec:funktionsweise-srg}

Die Shop-Rules-Generator-Komponente wartet über eine REST-Schnittstelle auf eine Anfrage und gibt für den angegebenen
Shop die Regeln für die Extraktion zurück.
Wenn die Regel für den Shop bereits in der Datenbank existiert, wird diese vom SRG zurückgegeben.
Andernfalls wird der Generierungsprozess für den Shop gestartet.

Während des Generierungsprozesses durchläuft der SRG mehrere Phasen.
Zuerst lädt der SRG eine bestimmte Anzahl von Angeboten von der idealo-Datenbank.
Dazu nutzt er die von idealo zur Verfügung gestellte IdealoBridge - dies ist eine API, welche den Zugriff über
eine REST-Schnittstelle kapselt.
Zu jedem dieser Angebote liegen die Adresse für das Angebot, sowie die  Informationen über die in
Kapitel~\ref{subsec:technische-anforderungen-parser} genannten Attribute vor.
Für jedes Angebot ruft der SRG die verlinkte Angebotsseite auf und speichert das HTML-Dokument dieser.
Wir wissen nun, dass es sich bei dem HTML-Dokument um das bestimmte Angebot handelt und können dies ausnutzen.
Dazu sucht der SRG den Wert aller einzelnen Produktattribute und erstellt für jeden Fundort eine Regel, welche auf
das entsprechende Element zeigt.
Nachdem er dies für alle heruntergeladenen Angebote gemacht hat, kann er aus den gesammelten Regeln eine
finale Regelmenge bestimmen.
Diese Regelmenge wird in einer Datenbank gespeichert und bei zukünftigen Anfragen direkt zurückgegeben.
    \newpage
    \section{Die Untersuchung der Genauigkeit der Selektoren}
\label{sec:evaluierung}

In diesem Kapitel wird untersucht, wie gut der Extraktionsalgorithmus funktioniert.
Dazu wird zunächst die Qualität der idealo-Daten, welche zum Anlernen und zum Evaluieren genutzt wird, grob analysiert.
Mit Hilfe der durchgeführten Messungen werden abschließend die Auswirkungen der Ergebnisse auf das Matching abgeleitet.

\subsection{Die Testdaten}
\label{subsec:testdaten}

Sowohl für das Antrainieren der Regeln als auch für das Evaluieren des Parsers werden Daten von idealo genutzt.
Für die nachfolgenden Messungen wurden Angebote von 50 Shops verwendet.
Für jeden Shop wurden 150 Angebote geladen.
Maximal 50 der 150 Angebote werden für das Anlernen verwendet.
Die Evaluation findet mit den restlichen 100 Angeboten statt.

Nachfolgend soll ein grober, quantitativer Überblick über die idealo-Daten gegeben werden.

\begin{figure}[H]
    \centering
    \includegraphics[width=\textwidth, trim=0 3cm 0 3cm, clip]{resources/Attributabdeckung-idealo-Daten.pdf}
    \caption{Attributabdeckung der idealo-Daten}
    \label{abb:testdaten}
\end{figure}

Der Abbildung~\ref{abb:testdaten} kann man entnehmen, dass es für jedes untersuchte Angebot einen Titel, einen Preis
und die SKU gibt.
Interessanterweise sind andere Produkteigenschaften, welche für die eindeutige Zuordnung genutzt werden können,
häufiger nicht vorhanden.
Der Verlauf der Attributabdeckung ähnelt sich in allen Datenreihen.
Die Verteilungen der Attribute innerhalb des Test- und des Trainingsset sind also ähnlich.
    \newpage
    \section{Das Fazit und der Ausblick}
\label{sec:abschluss}

Der Preisvergleich ist ein wichtiges Instrument, um die Markttransparenz im Internet zu gewährleisten.
Damit ein möglichst objektiver Preisvergleich sichergestellt werden kann, ist es erforderlich, einen annähernd
vollständigen Angebotskatalog zu vergleichen.
Das Ziel des Softwaresystems \textit{Scout} ist es, die Vollständigkeit des idealo-Angebotskatalogs zu untersuchen.

Für das Erfassen aller Produkte eines Onlinehändlers stellt die Datenextraktion einen wichtigen und schwierigen
Schritt dar.
Zur Lösung dieses Problems haben wir eine shop-spezifische Lösung entwickelt.
Die Evaluierung hat ergeben, dass der Algorithmus je nach Anwendungsfall eine hohe Präzision erreichen kann.
Auf den Testdaten von idealo wurde beispielsweise zwar nur jedes zweite Attribut gefunden, dafür jedoch auch eine
Präzision von über 95\% erreicht.
Die Ergebnisse können nun von der Matcher-Komponente des Softwaresystems Scout für den ähnlichkeitsbasierten
Vergleich verwendet werden.
\\
~\\
Das Projektpartner idealo ist bereits aktiv dabei, die implementierte Lösung aktiv in deren System zu
integrieren und weiter zu verbessern.
Für die zukünftige Weiterentwicklung besteht noch Potenzial bei der Entwicklung weiterer Selektoren, damit der Parser
weniger von der konkreten HTML-Struktur abhängig ist.
Des Weiteren kann man untersuchen, ob die Qualität der Selektoren durch eine gezielte Auswahl der Angebote, welche
für das Anlernen verwendet werden, verbessert werden kann.
Die Struktur von Shops kann sich unter Umständen innerhalb eines bestimmten Zeitraumes ändern.
Eine mögliche Lösung hierfür wäre zum Beispiel die Einführung einer \textit{Lebensdauer} für Regeln, damit diese nach
einer bestimmten Zeit neu generiert werden.
Abschließend kann man das Anlernen weiter verbessern, indem man die Normalisierungen der Angebotsdaten von idealo
rückgängig macht und an das Format der Shops anpasst.
Ein konkretes Beispiel wäre für den Preis nicht nur nach 12,00 zu suchen, sondern auch nach 12,0.
    \newpage

    \bibliography{main}
    \bibliographystyle{plaindin}      % BibTeX Styles nach Norm DIN 1505

    \newpage
    \section*{Die Selbstständigkeitserklärung}
\addcontentsline{toc}{section}{Die Selbstständigkeitserklärung}
\label{sec:selbstständigkeitserklärung}

Hiermit erkläre ich, die vorliegende Bachelorarbeit selbstständig erarbeitet und angefertigt, keine anderen als die
angegebenen Quellen und Hilfsmittel genutzt und jegliche Zitate sowie Verwendungen fremder Inhalte kenntlich gemacht
zu haben.
\vspace{0.5 cm}
\bigskip \\
Potsdam, 20.07.2018
\vspace{1 cm}
\begin{flushleft}
    \begin{tabular}{@{}L{7cm}}
        \dotfill \\
        ~Leonardo Hübscher
    \end{tabular}
\end{flushleft}

\end{document}
