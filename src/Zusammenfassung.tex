\section*{Zusammenfassung}
\markboth{Zusammenfassung}{Zusammenfassung}
\label{sec:abstract}

Durch die Vielzahl von Onlineshops und Fülle an Angeboten verliert der Onlinekäufer schnell die Übersicht.
Preisvergleichsplattformen wie idealo helfen dem Kunden das günstigste Angebot im Netz zu finden.
Die Gewährleistung der möglichst vollständigen Markttransparenz ist eine grundlegende Herausforderung für idealo.
Das von uns entwickelte Softwaresystem \textit{Scout} soll dabei helfen, den Produktkatalog von idealo auf
Vollständigkeit zu überprüfen und fehlende Angebote aufzulisten.
Ein wichtiger Prozessschritt ist dabei die Extrahierung von Produktinformationen, wie Produktname oder Preis, aus den
einzelnen Webseiten.
Die Schwierigkeit der Extraktion liegt darin, dass jeder Shop einen individuellen Aufbau besitzt und unterschiedlich
strukturiert ist.

Das entwickelte Parser-Modul löst dieses Problem, indem es für jeden Shop eigene Regeln zur Extraktion der
Produktinformationen verwendet.
Dabei ist es nicht erforderlich, dass diese Regeln manuell erstellt werden müssen.
Durch die Nutzung der bereits vorhanden Angebote aus dem Bestand von idealo kann die Extrahierung der Struktur
mittels maschinellen Lernens erfasst werden.
Durch die Erhöhung der Flexibilität der Datenextraktion und durch die Einführung eines Bewertungssystems konnte der
Extraktionsalgorithmus weiter verbessert werden.
Messungen, welche auf 50 verschiedenen Shops basieren, haben ergeben, dass die Produktinformationen mit einer
Genauigkeit von über 95 Prozent bei einer Trefferquote von etwa 50\% extrahiert werden können.