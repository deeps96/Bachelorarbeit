\section{Die Welt der Preisvergleichsportale}
\label{sec:einleitung}

Der Fernhandel ist bereits seit der Steinzeit ein wichtiger Bestandteil der Gesellschaft.
Durch die rasante Entwicklung des Internets und der steigenden Anzahl der Onlinehändler vergrößert sich das
Produktangebot.\\
Heutzutage kann ein Käufer aus einer Vielzahl von Artikeln wählen und muss sich nicht wie in der Steinzeit auf
einen Händler oder auf die lokale Verfügbarkeit beschränken.

\subsection{Der Onlinehandel von heute}
\label{subsec:onlinehandel-heute}

In den letzten Jahren hat der Onlinehandel sowohl an Bedeutung für die Unternehmen, als auch für die Kunden gewonnen.
Laut einer Statistik von Eurostat machte im Jahr 2017 der Onlinehandel 21\% des Gesamtumsatzes deutscher Unternehmen
aus und stellte somit einen nicht unerheblichen Anteil dar~\cite{statista:anteil-gesamtumsatz-europa}.
Aus einer weiteren Statistik geht hervor, dass 2017 zwei Drittel der Deutschen regelmäßig online
einkauften~\cite{statista:anteil-online-kaeufer-europa}.

Die Entwicklung bringt jedoch ein Problem mit sich: Mit der steigenden Auswahl an Produkten verliert ein
potenzieller Käufer schnell die Übersicht.
Preisvergleichsportale versuchen deshalb die Markttransparenz wiederherzustellen.
Ein Käufer soll sich sicher sein das für ihn beste Angebot zu finden.

Das Angebot der Vergleichsportale wird von den Internetnutzern gut aufgenommen, wie eine Statistik aus 2017 zeigt:
Für einen besseren Vergleich nutzen rund zwei Drittel die Möglichkeit sich im Internet zum Produkt oder zum
Preis zu informieren~\cite{statista:internetnutzer-preisvergleich-deutschland,
statista:anteil-online-kaeufe-deutschland}.

In einem vom Nachrichtensender n-tv beauftragten Test\footnotemark, hat das Deutsche Institut für Service-Qualität
mehrere Preissuchmaschinen unter dem Aspekt des günstigsten Preises, der Preisaktualität und dem Nutzererlebnis
verglichen.
\footnotetext{https://disq.de/2014/20141004-Preissuchmaschinen.html}
\newline
Im Ergebnis hat idealo.de in allen Kategorien den ersten Platz eingenommen, gefolgt von billiger.de und preis.de.
Es wurde jedoch bemängelt, dass selbst beim besten Preisvergleich nur für die Hälfte der Produkte der günstigste
Shop angezeigt wurde.

\subsection{Das Preisvergleichsportal idealo}
\label{subsec:idealo}

Die Mission des Preisvergleichsportals idealo ist es, den Vergleich für den Kunden stetig zu verbessern.
Je mehr Angebote idealo vergleicht, desto sicherer kann sich der Kunde sein, das tatsächlich günstigste Angebot zu
finden.

Dazu schließt idealo Verträge mit mehreren Onlinehändlern ab.
Diese verpflichten sich, Daten zu ihren Angeboten an idealo zu übermitteln und zu aktualisieren.
Für jeden vermittelten Kauf zahlen die Shops an idealo eine Provision.
Diese basiert auf CPC (Kosten pro Klick) oder CPO (Kosten pro tatsächlicher Bestellung).

Um auch zukünftig wettbewerbsfähig zu bleiben arbeitet idealo daran, auch für die letzten 50\% der Produkte
immer das beste Angebot liefern zu können.
Dies erreicht es zum einen durch Vertragsabschlüsse mit weiteren Onlineshops und zum anderen durch die
Sicherstellung, dass tatsächlich alle Angebote eines Vertragspartners gelistet werden.

\subsection{Das Ziel des Bachelorprojektes}
\label{subsec:projektziel}

Es soll eine Software entworfen werden, welche eine automatisierte Bestandsanalyse für einen bestimmten
Vertragspartner durchführt.
Mit Hilfe des resultierenden Berichtes soll es möglich sein, herauszufinden welche Angebote des Vertragspartner im
Produktkatalog von idealo fehlen.
\\
~\\
Der Bericht soll Informationen darüber enthalten, welche Produkte nicht vorhanden sind, aus welcher Kategorie diese
stammen und zu welcher Preisregion die Produkte gehören.
Durch diese Übersicht soll ein Mitarbeiter von idealo dazu befähigt werden, die Ursachen für das Fehlen der Angebote
herauszufinden.

idealo vermutet, dass ein unbeabsichtigtes Fehlen von Produkten durch einen fehlerhaften Importvorgang zu erklären ist.
Zudem könnte es sein, dass ein Händler bewusst nicht alle Produkte bei idealo führen möchte.
\\
~\\
Für die Entwicklung dieser Lösung hatten wir als fünfköpfiges Team neun Monate Zeit.
Zudem wurde uns ein Betreuer von idealo zur Verfügung gestellt, welcher die funktionalen Anforderungen an die
Software kommunizierte und als technischer Berater diente.
Er begleitete uns während des gesamten Entwicklungsprozesses und unterstützte uns bei Fragen bezüglich der
Systemarchitektur.

\subsection{Die Microservice-Architektur des Scout-Softwaresystems}
\label{subsec:microservice-architektur}

Wir haben uns dafür entschieden, das Gesamtsystem als Microservice-Architektur zu konzipieren.

Die Mircoservice-Architektur ermöglicht es logisch gekapselte Komponenten zu entwickeln, welche sich sehr gut
skalieren und erweitern lassen.
Eine ausführlichere Begründung für diese Architekturentscheidung kann in der Bachelorarbeit von Dmitrii
Zhamanakov nachgeschlagen werden.\cite{theses:dmitrii}

Für die Implementierung der Architektur haben wir die Programmiersprache Java gewählt und verwenden diese in
Kombination mit dem Spring-Framework\footnotemark.
\footnotetext{\url{https://spring.io/}}

Das entwickelte Gesamtsystem \textit{Scout} besteht grob gesehen aus drei Komponenten: dem Crawler, dem Parser und dem
Matcher.

Der \textit{Crawler} ist für das Herunterladen jeder einzelnen Seite eines Shops verantwortlich.
Jonas Pohlmann hat sich im Projektverlauf intensiv mit verschiedenen Crawling-Frameworks auseinandergesetzt und diese
in seiner Bachelorarbeit verglichen.\cite{thesis:jonas}

Die Funktionsweise der maschinenlernbasierten \textit{Matcher}-Komponente wird in der Bachelorarbeit von Tom
Schwarzburg näher beschrieben.\cite{thesis:tom}
Der Matcher vergleicht die vom System gefundenen Angebote mit den Angeboten, die idealo bereits kennt.

Damit dieser die geladenen Angebote mit dem Katalog von idealo vergleichen kann, muss das heruntergeladene
HTML-Dokument in ein Format gebracht werden, welches der Computer für den Vergleich nutzen kann.

Diesen Schritt erledigt der Parser, welcher zwischen Crawler und Matcher agiert.
Der Fokus dieser Arbeit liegt in der Beschreibung der Funktionsweise und des Aufbaus des Parsers.