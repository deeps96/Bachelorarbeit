\section{Einleitung}
\label{sec:einleitung}

Der Fernhandel ist bereits seit der Steinzeit ein wichtiger Bestandteil der Gesellschaft.
Durch die rasante Entwicklung des Internets und der steigenden Anzahl der Onlinehändler vergrößert sich das
Produktangebot.
Der moderne Kunde kann aus einer Vielzahl von Artikeln wählen und muss sich nicht wie in der Steinzeit auf einen
Händler oder auf die lokale Verfügbarkeit beschränken.

\subsection{Der Onlinehandel von heute}
\label{subsec:onlinehandel-heute}

\begin{comment}
    https://de.statista.com/statistik/studie/id/23510/dokument/e-commerce-in-europa-statista-dossier/
    https://de.statista.com/statistik/studie/id/6558/dokument/produktvergleich-im-internet-statista-dossier/
\end{comment}


In den letzten Jahren hat der Onlinehandel sowohl an Bedeutung für die Unternehmen, als auch für die Kunden gewonnen.
Laut einer Statistik von Eurostat machte im Jahr 2017 der Onlinehandel 21\% des Gesamtumsatzes deutscher Unternehmen
aus und stellt somit einen nicht unerheblichen Anteil dar.
Aus einer weiteren Statistik geht hervor, dass 2016 zwei Drittel der Deutschen regelmäßig online einkauften.

Die Entwicklung bringt jedoch ein Problem mit sich: Mit der steigenden Auswahl an Produkten verliert ein
potenzieller Käufer schnell die Übersicht.
Preisvergleichsportale versuchen deshalb, die Markttransparenz wiederherzustellen.
Ein Käufer soll sich sicher sein, das für ihn beste Angebot zu finden.
Das Angebot der Vergleichsportale wird von den Internetnutzern gut aufgenommen, wie eine Statistik aus 2017 zeigt:
Rund zwei Drittel nutzen die Möglichkeit, sich im Internet weitere Informationen zum Produkt oder zum Preis für einen
besseren Vergleich zu holen.

In einem von n-tv beauftragten Test, hat das Deutsche Institut für Service-Qualität mehrere Preissuchmaschinen unter
dem Aspekt des günstigsten Preises, der Preisaktualität und dem Nutzererlebnis verglichen.
Im Ergebnis hat idealo.de in allen Kategorien den ersten Platz eingenommen, gefolgt von billiger.de und preis.de.
Bemängelt wurde jedoch, dass selbst beim besten Preisvergleich nur für die Hälfte der Produkte der günstigste Shop
angezeigt wurde.

\subsection{Das Preisvergleichsportal idealo}
\label{subsec:idealo}

Die Aufgabe des Preisvergleichsportals idealo ist es, den Vergleich für den Kunden stetig zu verbessern.
Je mehr Angebote idealo vergleicht, desto sicherer kann sich der Kunde sein, das tatsächlich günstigste Angebot zu
finden.
Dazu hat idealo Verträge mit mehreren Onlinehändlern, welche sich verpflichten, Daten zu ihren Angeboten an idealo zu
übermitteln und zu aktualisieren.
Für jeden vermittelten Kauf erhält idealo von den Shops eine kleine Provision.
Diese basiert auf CPC (Kosten pro Klick) oder CPO (Kosten pro tatsächlicher Bestellung).

Um auch zukünftig wettbewerbsfähig zu bleiben, arbeitet idealo daran auch für die letzten 50\% der Produkte,
immer das beste Angebot liefern zu können.
Dies erreicht es zum einen durch das Abschließen von Verträgen mit weiteren Onlineshops.
Zum anderen möchte idealo jedoch auch sicherstellen, tatsächlich alle Angebote eines Vertragspartners zu listen.

\subsection{Das Projektziel}
\label{subsec:projektziel}

Die Aufgabe des Projektes besteht darin, für idealo herauszufinden, welche Angebote der Vertragspartner sich nicht im
Produktkatalog von idealo befinden.
Es soll eine Software entworfen werden, mit Hilfe deren ein Mitarbeiter für einen beliebigen Vertragspartner einen
solchen Bericht enthält.
Dieser soll Informationen darüber enthalten, welche Produkte fehlen, aus welcher Kategorie diese stammen und zu
welcher Preisregion die Produkte gehören.
Eine Messung eines Shops soll dabei in einem zeitlich akzeptablen Rahmen durchführbar und wiederholbar sein.
Durch diesen Bericht soll der Mitarbeiter dazu befähigt werden, die Ursachen für das Fehlen der Angebote herauszufinden.
idealo vermutet, dass ein unbeabsichtigtes Fehlen von Produkten durch einen fehlerhaften Importvorgang zu erklären ist.
Zudem könnte es sein, dass ein Händler bewusst nicht alle Produkte bei idealo führen möchte.

Für die Entwicklung dieser Lösung hatten wir als fünfköpfiges Team sechs Monate Zeit.
Zudem wurde uns ein Betreuer von idealo zur Verfügung gestellt.
Dieser diente als Kommunikationsschnittstelle für den Austausch der funktionalen Anforderungen und als technischer
Berater.
Er begleitete uns während des gesamten Entwicklungsprozesses und unterstützte uns bei Fragen bezüglich der
Systemarchitektur.

\subsection{Die nichtfunktionalen Anforderungen}
\label{subsec:nichtfunktionale-anforderungen}

Zu Beginn des Projektes wurden die nichtfunktionalen Anforderungen festgelegt.
Für den gewinnbringenden Einsatz war es idealo besonders wichtig, dass das System für möglichst alle Shops einsetzbar
ist.
Die allgemeinen Herausforderungen an das zu entwickelnde Softwaresystem sind vielseitig.
Mit ca. 5000 Shops und über zwei Millionen Angeboten in Deutschland gibt es sehr viele Daten, welche verarbeitet
werden müssen.
Das System soll daher skalierbar, hochgradig parallel, effizient und robust sein.
idealo möchte hierbei nicht das Tagesgeschäft der Shops beeinflussen und erwartet, dass dies bei der Umsetzung der
Lösung berücksichtigt wird.

Um den entwickelten Code später abändern oder anderweitig weiterverwenden zu können, soll dieser einfach zu
verstehen, erweiterbar und gut dokumentiert sein.
Diese Anforderungen hatten Auswirkungen auf die Entwicklung der gesamten Architektur, sowie der einzelnen Komponenten.

\subsection{Die zugrundeliegende Microservice-Architektur}
\label{subsec:microservice-architektur}

Ausgehend von den nichtfunktionalen Anforderungen haben wir uns dafür entschieden, das Gesamtsystem als
Microservice-Architektur zu konzipieren.
Die Mircoservice-Architektur ermöglicht es, logisch gekapselte Komponenten zu entwickeln, welche sich sehr gut
skalieren und erweitern lassen.
Durch die Verwendung eines Load-Balancers, kann die Last im Gesamtsystem optimal verteilt werden.
Des Weiteren wird Continious Deployment, also das kontinuierliche Integrieren der aktuellsten Komponenten vereinfacht.

Für die Implementierung der Architektur haben wir die Programmiersprache Java gewählt.
Wir haben uns für diese Sprache auf Grundlage der Anforderung, dass die Software in einer virtuellen Maschine laufen
soll, entschieden.
Für die Umsetzung der Microservices und deren REST-Schnittstellen verwenden wir das Spring-Framework, welches die
Anbindung an verschiedenste Datenquellen und -senken vereinfacht.

Das entwickelte Gesamtsystem besteht grob gesehen aus drei Komponenten: dem Crawler, dem Parser und dem Matcher.
Der mögliche Datenfluss sieht dabei wie folgt aus:
Ein idealo-Mitarbeiter fragt den Bericht für einen bestimmten Shop an.
Wenn der Bericht noch nicht existiert, beginnt als erstes die Crawler-Komponente jede einzelne Seite des Shops
herunterzuladen.
Damit der Matcher die geladenen Angebote mit dem Katalog von idealo vergleichen kann, muss das heruntergeladene
HTML-Dokument in ein Format gebracht werden, welches der Computer für den Vergleich nutzen kann.
Dieser Schritt erledigt der Parser, welcher zwischen Crawler und Matcher agiert.
Der Parser ist dafür verantwortlich, die relevanten Produktinformationen aus den HTML-Dateien zu extrahieren und zu
normalisieren.
Die extrahierten Informationen werden anschließend von dem Matcher für das Vergleichen verwendet.
Dieser erstellt abschließend einen Bericht, der von dem Mitarbeiter abgerufen werden kann.

Der Fokus dieser Arbeit liegt in der Beschreibung der konkreten Funktionsweise der Parser-Komponente.

\subsection{Die technischen Anforderungen an den Parser}
\label{subsec:technische-anforderungen-parser}

Das Extrahieren der Informationen ist ein sehr wichtiger und kritischer Schritt, da die Genauigkeit die Ergebnisse
des Matchers und somit auch des Gesamtergebnisses stark beeinflusst.
Fehlerhafte Informationen können das Ergebnis so stark verfälschen, dass die Aussagekraft des Gesamtergebnisses des
Systems in Frage gestellt werden könnte.

Die Herausforderung der Parser-Komponente besteht hauptsächlich darin, das heterogene Informationsschemata der
verschiedenen Shops in ein homogenes, normalisierten Schema zu bringen.
Im Detail geht es darum, zu jedem Angebot den Titel, die Produktbeschreibung, den Preis, die Marke, die Kategorie,
die Produktbilder sowie weitere eindeutige Merkmale im Format von idealo zu erfassen.
Diese eindeutigen Merkmale sind zum Beispiel die standardisierte EAN (Europäische Artikelnummer), HAN (Händler
Artikelnummer) und SKU (Stock keeping unit/ eine shop-spezifische Kennung).

Da die Crawler-Komponente sehr viel Zeit benötigt um alle Seiten zu erfassen, spielt der Zeitfaktor für den Parser
keine große Rolle.
Eine schnelle Verarbeitung der Seiten ist dennoch wünschenswert.
Damit die Ergebnisse des Parsers als zuverlässig eingestuft werden und vom Matcher weiterverwenden werten können, ist
eine hohe Präzision und eine hohe Extraktionsrate erforderlich.
