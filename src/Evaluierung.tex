\section{Die Untersuchung der Genauigkeit der Selektoren}
\label{sec:evaluierung}

In diesem Kapitel wird untersucht, wie gut der Extraktionsalgorithmus funktioniert.
Dazu wird zunächst die Qualität der idealo-Daten, welche zum Anlernen und zum Evaluieren genutzt wird, grob analysiert.
Mit Hilfe der durchgeführten Messungen werden abschließend die Auswirkungen der Ergebnisse auf das Matching
abgeleitet.\\
\newline
Sowohl für das Antrainieren der Extraktionsregeln als auch für die Evaluierung der Ergebnisse wurden Angebotsdaten
von idealo verwendet.
Für die Messung wurden 7500 Angebote von 50 Shops genutzt.
Je Shop wurden max.\ 50 Angebote für die Erstellung der Regeln und 100 Angebote für die Evaluierung als Testmenge
verwendet.
Die Auswahl der Shops erfolgte basierend auf der Shopübersicht\footnotemark von idealo.
\footnotetext{https://www.idealo.de/preisvergleich/AllePartner.html}
Für jedes Angebot existiert mindestens der Titel, der Preis und die SKU.
Alle nachfolgenden Messungen basieren auf einem Schnappschuss der Angebotsdaten, sowie der verlinkten Angebotswebseiten.
Die Links wurden nicht durch die URL-Cleaner-Komponente bereinigt, da sonst die Gefahr besteht, dass die
Angebotsdaten von idealo mit denen von der Webseite nicht übereinstimmen.
Deshalb wurden nicht mehr als 150 Angebote je Shop heruntergeladen.

Die Genauigkeit des Parser wurde bestimmt, indem die Extraktionsregeln auf der Testmenge angewandt wurden.
Die Genauigkeit wurde dabei in Abhängigkeit der Menge der Angebote (SaS), welche für das Anlernen verwendet wurden,
sowie des in Kapitel~\ref{subsec:bewertungssystem} eingeführten Filter-Schwellwertes (F) bestimmt.
In der Tabelle~\ref{tab:accuracy-precision} sind die Genauigkeit und die Präzision aufgeführt.
Die Genauigkeit gibt an, wie oft ein Produktattribut korrekt erfasst wurde.
Die Präzision berücksichtigt zusätzlich die Fälle, in denen der Parser explizit nichts zurück gibt.

\begin{table}[h]
    \centering
    \begin{tabular}{ c | c c c | c c c }
        &   \multicolumn{3}{c}{\textit{Genauigkeit in \%}}    &   \multicolumn{3}{c}{\textit{Präzision in \%}} \\
        \textbf{F\textbackslash SaS} & \textbf{10} & \textbf{20} & \textbf{50} & \textbf{10} & \textbf{20} & \textbf{50}  \\
        \hline
        \textbf{0}       &   50.59 &   50.78 &   51.07         &   72.73 &   70.81 &   69.62 \\
        \textbf{0.5}     &   52.12 &   53.50 &   \textbf{53.98}&   88.66 &   90.39 &   91.22 \\
        \textbf{0.6}     &   51.87 &   53.07 &   53.15         &   94.15 &   94.03 &   94.93 \\
        \textbf{0.7}     &   50.15 &   51.37 &   52.02         &   96.15 &   96.39 &   95.86 \\
        \textbf{0.8}     &   47.92 &   49.69 &   50.05         &   97.84 &   97.81 &   97.76 \\
        \textbf{0.9}     &   44.61 &   46.92 &   46.57         &   98.16 &   98.14 &   98.42 \\
        \textbf{1.0}     &   41.48 &   39.27 &   36.00         &   98.17 &   97.64 &   \textbf{98.90}

    \end{tabular}
    \caption{Genauigkeit und Präzision bei unterschiedlichen Konfigurationen}
    \label{tab:accuracy-precision}
\end{table}

