\section{Die Extraktion produktspezifischer Daten}



- Erläuterung der zugrundeliegenden Idee:
	- Ausnutzen, dass Shops CMS nutzen und Angebote eines einzelnen Shops ähnliche Struktur haben
	- Idealo Datensatz nutzen, um Struktur zu erlernen
	- HTML -> sehr strukturiert -> man lernt Struktur für jeden Shop
\subsection{Die Funktionsweise des Extraktionsalgorithmus}

	- Untergliederung in 3 Komponenten -> Parser, SRG und URLCleaner anhand Herleitung
	- Klärung/ Definition von Begriffen: Rule, Selector, NodeTypes
	- Erklärung des Datenflusses zwischen den Komponenten in groben Schritten (Dataflow)
	- besuchen von IdealoOffer-Links erforderlich -> warum braucht man URL-Cleaner?
	
\subsection{Das Erstellen von shop-spezifischen Regeln}

	- High Level erklären, was die ShopRulesGenerator KLASSE macht
	- welche Schritte werden durchlaufen? -> evntl. bietrt sich hier auch PSEUDO-Code an
	- es werden alle occurrences gesucht und je nach NodeType wird spezieller Generator verwendet
	-> es gibt je ein Generator pro NodeType	
	
\subsection{Das Bereinigen von Links mit Trackerinformationen}
	- RedirectClean und TrackerClean erklären
	- nun können Seiten besucht und Regeln generiert werden	
	
\subsubsection{Das Erstellen von Regeln für TextNodes}

	- Einfachste Selektor: CSS Selektor

\subsubsection{Das Erstellen von Regeln für AttributeNodes}

	- Weiterentwicklung des TextNodeSelektors
	- Zusätzlicher Selektor (non-numeric)

\subsubsection{Das Erstellen von Regeln für DataNodes}

	- Idee
	- woraus besteht der Selektor?
	- Pseudo-Code?

\subsection{Das Bewertungssystem für Selectors}

	- Wie funktioniert es?
	- Warum ist "Müll" zu finden schlimmer als nichts zu finden? -> Weil man "nichts" von richtig unterscheiden kann

\subsection{Allgemeine Verbesserungen für alle Selector-Arten}

	- EAN -> Generische Regel beim Parser
	- Left cut/ right cut
	
\subsection{Die Implementierung des Extraktionsalgorithmus}

	- Welche Sprache, Frameworks, Dependencies wurden verwendet?
	- Gab es Probleme? -> Flexibilität HTML, Javascript/ JSON
	
\subsection{Weitere Herausforderungen}
	
	- Was sind noch offene Verbesserungen? -> siehe README auf Github

\subsection{Kategorisierung der HTML-Komponenten}

	- Theoretischer Einschub, um die drei Komponentenarten zu erklären:
	- TextNode
	- AttributeNode
	- DataNode