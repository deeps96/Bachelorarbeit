\section{Der Abschluss}

Der Preisvergleich ist ein wichtiges Instrument, um die Markttransparenz im Internet zu gewährleisten.
Damit ein möglichst objektiver Preisvergleich sichergestellt werden kann, ist es erforderlich, einen möglichst
vollständigen Angebotskatalog zu vergleichen.
Das Ziel des Softwaresystems \textit{Scout} ist es, die Vollständigkeit des idealo-Angebotskatalogs zu untersuchen.

Für das Erfassen aller Produkte eines Onlinehändlers stellt die Datenextraktion einen wichtigen und schwierigen
Schritt dar.
In dieser Arbeit wurden zwei mögliche Herangehensweisen für das Extrahieren der Produktattribute vorgestellt.
Der shop-spezifische Ansatz produziert im Vergleich zum shop-unspezifischen als zuverlässigere Lösung herausgestellt.
Die Evaluierung hat ergeben, dass der Algorithmus je nach Anwendungsfall eine hohe Präzision erreichen kann.
Auf den Testdaten von idealo wurde zwar nur jedes zweite Attribut gefunden, dafür jedoch auch eine Präzision von über
95\% erreicht.
\\
~\\
Für die zukünftige Weiterentwicklung besteht noch Potenzial bei der Entwicklung weiterer Selektoren, um die
Flexibilität des Parser weiter zu erhöhen.
Des Weiteren kann man untersuchen, ob die Qualität der Selektoren durch eine gezielte Auswahl der Angebote, welche
für das Anlernen verwendet werden, verbessert werden kann.

Abschließend möchte ich auf ein Problem bei der technischen Umsetzung im Zusammenhang mit der Bibliothek
Jsoup\footnotemark hinweisen: Für eine produktive Weiterentwicklung sollte man evaluieren, wie man eine höhere
Flexibilität bei fehlerhaften HTML-Dokumenten gewährleisten könnte.
\footnotetext{https://jsoup.org}