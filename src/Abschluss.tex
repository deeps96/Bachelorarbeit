\section{Der Ausblick und das Fazit}
\label{sec:abschluss}

Das Projektpartner idealo ist bereits aktiv dabei, die implementierte Lösung aktiv in deren System zu
integrieren und weiter zu verbessern.
Für die zukünftige Weiterentwicklung besteht noch Potenzial bei der Entwicklung weiterer Selektoren, damit der Parser
weniger von der konkreten HTML-Struktur abhängig ist.
Des Weiteren kann man untersuchen, ob die Qualität der Selektoren durch eine gezielte Auswahl der Angebote, welche
für das Anlernen verwendet werden, verbessert werden kann.
Die Struktur von Shops kann sich unter Umständen innerhalb eines bestimmten Zeitraumes ändern.
Eine mögliche Lösung hierfür wäre zum Beispiel die Einführung einer \textit{Lebensdauer} für Regeln, damit diese nach
einer bestimmten Zeit neu generiert werden.
Abschließend kann man das Anlernen weiter verbessern, indem man die Normalisierungen der Angebotsdaten von idealo
rückgängig macht und an das Format der Shops anpasst.
Ein konkretes Beispiel wäre für den Preis nicht nur nach 12,00 zu suchen, sondern auch nach 12,0.

Der Preisvergleich ist ein wichtiges Instrument, um die Markttransparenz im Internet zu gewährleisten.
Damit ein möglichst objektiver Preisvergleich sichergestellt werden kann, ist es erforderlich, einen annähernd
vollständigen Angebotskatalog zu vergleichen.
Das Ziel des Softwaresystems \textit{Scout} ist es, die Vollständigkeit des idealo-Angebotskatalogs zu untersuchen.
Für das Erfassen aller Produkte eines Onlinehändlers stellt die Datenextraktion einen wichtigen und schwierigen
Schritt dar.
Zur Lösung dieses Problems haben wir eine shop-spezifische Lösung entwickelt.
Die Evaluierung hat ergeben, dass der Algorithmus je nach Anwendungsfall eine hohe Präzision erreichen kann.
Auf den Testdaten von idealo wurde beispielsweise zwar nur jedes zweite Attribut gefunden, dafür jedoch auch eine
Präzision von über 95\% erreicht.
Die Ergebnisse können nun von der Matcher-Komponente des Softwaresystems Scout für den ähnlichkeitsbasierten
Vergleich verwendet werden.